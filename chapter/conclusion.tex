\documentclass[/Users/ikedahajime/GitHub/reserch/master_report/thesis]{subfiles}
% このファイル内だけのコマンド
\begin{document}
\chapter{結論と展望}
本研究では、円の中に閉じ込められた ABP 及び CABP についてシミュレーションを行い、
それらの流れや渦について調べた。

まず ABP 系において低密度領域、及び高密度領域について流れの慣性、 Péclet 数依存性を調べた。
低密度領域においては、慣性の小さな領域では Péclet 数を大きくすると粒子の壁への凝集が起こったが、慣性を大きくすると
粒子の凝集はなくなり、一様な状態になった。また、慣性の大きな領域では系全体の流れは存在しなかった。

高密度系についても流れを調べた。流れについて、定性的には慣性の大きさによる変化はないことが分かった。
また、 Péclet 数を大きくすると系全体に流れが生じた。この現象は先行研究である1次元系\cite{capriniCollectiveEffectsConfined2021}
と同様の現象である。この流れの時間依存性についても観測し、全角運動量の時間依存性は ABP の1粒子における
時間変化で説明できることがわかった。


次に、CABP 系においては、高密度系において粒子の回転半径 $R_\Omega$ を変化させて同様に流れや渦について調べた。
$R_\Omega$ が大きくなるに従って系は空間的に無秩序になる無秩序相、
粒子全ての運動方向が揃い、規則正しく振動する振動相、全ての粒子が1つの渦となって回り続ける定流相へと変化した。
これらについての相図を作成し、
無秩序相と振動相の間の遷移は速度相関の相関長と系の半径が等しくなる領域で起きていることが分かった。これは相関長が円の半径と等しい時に、
円が渦を切り取っていることを示す。振動相と定流相の間の遷移は円形のチューブに閉じ込められた1粒子描像で説明でき、
1粒子 CABP の実行的な速度が小さくなった状態であると考えられる。

また、無秩序相において系の中に複数の渦が発生した。これらの渦は速度相関の相関長と系の半径によってスケールすることができ、
この渦についても相関長によって説明することができる。

以上の結果は、自己駆動力による速度相関と渦、閉じ込めによる効果を明らかにするものであり、
閉じ込め系における渦形成は相関長によって起こることを示している。

本研究の今後の課題として、渦と欠陥の関係が挙げられる。無秩序相の渦が複数発生する領域において$\phi_6$の欠陥は
見られず、このことから渦の発生と欠陥は無相関である。
しかし、本研究では高密度領域である固相\cite{kurodaLongrangeTranslationalOrder2024}を探索しているため、この現象が
単に高密度であるために$\phi_6$の欠陥が発生しなかった可能性が存在する。そこで、液相を探索する
ことで渦の発生と欠陥との関係が明らかになることが期待される。


\end{document}
